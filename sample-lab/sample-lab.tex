\documentclass[11pt]{article}

\usepackage{../style/ece361}

%-------------------------------------------------------------------------------------------------

% Lab parameters
\def\thelab{0}
\def\datedue{Dec. 31, 1969 @ 7 PM}

\useCompactTitle{ECE361 Lab \thelab}{Sample Lab Document}

%-------------------------------------------------------------------------------------------------
\begin{document}
% Create title page, and force header and footer onto it
\maketitle \thispagestyle{fancy}

\hfill {\large \textbf{Due \datedue}}
%-------------------------------------------------------------------------------------------------


%-------------------------------------------------------------------------------------------------
\section{Overview}
\label{sec:overview}
%-------------------------------------------------------------------------------------------------
The ability to access and manipulate data over a networked connection has become a fundamental feature in most modern day applications and services. Chances are, if you are taking this course, you have likely never owned a computing device without a network connection. We live in an increasingly data-driven and networked society powered by applications and services that are constantly communicating data.

\think{You can use this box to ask thought-inducing questions. For example, can anyone guess the significance of the due date?}



%-------------------------------------------------------------------------------------------------
\section{Prelab}
\label{sec:prelab}
%-------------------------------------------------------------------------------------------------
Any prelab instructions go here.

\work{You can use this box to highlight a specific work item.\\
e.g. Finish the prelab \textbf{BEFORE} the lab.}

If the lab doesn't have any pre-lab, you can easily comment this section out.

%-------------------------------------------------------------------------------------------------
\section{Part 1: Fancy description here}
\label{sec:part1}
%-------------------------------------------------------------------------------------------------
Description of part 1 (or 2, or 3, etc.) here.

\hint{If you want to stress a hint for students, you can use this box.}

This part can be broken down into multiple sub-parts...

\subsection{Step one}
\label{subsec:part1-step1}
Printing out status and/or error messages in the code is a great way to debug.

\subsection{Step two}
\label{subsec:part1-step2}
Closing out part 1, enjoy this image of Spock.

% figvs takes 4 parameters (see the style sheet for source code)
%   1) Image width (w.r.t. column)
%   2) Image file name
%   3) Other 'includegraphics' parameters
%   4) Caption
\figvs{0.4}{spock}{trim=0cm 0cm 0cm 0cm,clip}{Spock from Amok Time. First aired Sept. 15th 1967.}

%-------------------------------------------------------------------------------------------------
\section{Part 2: Oh no! Some code!}
\label{sec:part2}
%-------------------------------------------------------------------------------------------------
We can even embed code snippets.
\begin{lstlisting}[caption={Some snippet of code}, language=Python]
hellostr = "Hello"
worldstr = "world!"

print hellostr + worldstr
\end{lstlisting}

\warn{Warning to students: don't divide by 0!}

%-------------------------------------------------------------------------------------------------
\section{TA Grading}
\label{sec:tagrading}
%-------------------------------------------------------------------------------------------------
Questions asked by TAs will be worth 20\% of your mark.

%-------------------------------------------------------------------------------------------------
\section{Exerciser}
\label{sec:exercise}
%-------------------------------------------------------------------------------------------------
To run the exerciser, lift up the PC and do 50 laps around Galbraith building.

%-------------------------------------------------------------------------------------------------
\section{Submission}
\label{sec:submission}
%-------------------------------------------------------------------------------------------------
Some submission instructions here.

\end{document}
