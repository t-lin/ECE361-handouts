\documentclass[11pt]{article}

\usepackage{../style/ece361}

%-------------------------------------------------------------------------------------------------

% Lab parameters
\def\thelab{4}
\def\datedue{Mar. 21, 2020 @ 11:59 PM}

\useCompactTitle{ECE361 Lab \thelab}{Medium Access Control (MAC)}

%-------------------------------------------------------------------------------------------------
\begin{document}
% Create title page, and force header and footer onto it
\maketitle \thispagestyle{fancy}

\hfill {\large \textbf{Due \datedue}}
%-------------------------------------------------------------------------------------------------


%-------------------------------------------------------------------------------------------------
\section{Overview}
\label{sec:overview}
%-------------------------------------------------------------------------------------------------
One of the fundamental design challenges in networking is the issue of sharing access to a single communication medium.
This challenge is apparent in wireless networks, but also exists in wired networks, as you’ll see in the second half of this lab.
Medium Access Control (MAC) protocols enable multiple stations to share access to a single medium; and the choice of protocol (and its various parameters) directly affects the efficiency of the channel utilization.

In this lab you will simulate two alternative Medium Access Control (MAC) protocols and explore their properties.
In the first part, you will be given a working simulator for the ALOHA MAC protocol ("Pure ALOHA") to explore, which you must modify to implement Slotted ALOHA.
In the second part of the lab, you will implement the core logic of a Carrier Sense Multiple Access with Collision Detection (CSMA/CD) simulator.


%-------------------------------------------------------------------------------------------------
\section{Part 1: Slotted ALOHA}
\label{sec:sotted-aloha}
%-------------------------------------------------------------------------------------------------
In this part, you will begin by taking the ALOHA simulator (given to you) and extend it to become Slotted ALOHA.
In Slotted ALOHA, stations are synchronized to transmit frames only at the start of a slot interval, and thus, it is simply a discretized version of ALOHA.

The code given to you takes a single command-line parameter, \texttt{G}, and runs the simulation based on that parameter.
The output of the simulation will be appended to a file called \texttt{out.txt}, where each line is a comma-delimited entry showing \texttt{G} and \texttt{S}, respectively.
When you run the simulation multiple times over, with varying values of \texttt{G}, you can populate the output file with a series of values \textbf{(see Lst. TODO)} which you can then graph \textbf{(see Fig. TODO)}.

\begin{lstlisting}[style=ece361shell, caption={Sample data from \texttt{out.txt} for ALOHA}, label={lst:aloha-output}]
0.03125,0.029624704097476643
0.0625,0.057893199350203525
0.125,0.10240504689659247
0.25,0.15737031469013515
0.5,0.19784254160579692
1.0,0.1651676868095355
2.0,0.04724612507409784
\end{lstlisting}

There are other parameters within the simulation code that you can alter, but for the purposes of the lab, we have hard-coded most of it for you.

The code given to you has already calculated the slot interval.
Study the code carefully, particularly when the stations decide to transmit a frame.
Your task is to discretize the simulator by syncing the frame transmissions to the start of each interval.

After you have made your modifications, generate a new G vs S curve and see if it matches the theoretical improvement of Slotted ALOHA over ALOHA \textbf{(see Fig. TODO)}.



%-------------------------------------------------------------------------------------------------
\section{Carrier Sense Multiple Access with Collision Detection (CSMA/CD)}
\label{sec:csma-cd}
%-------------------------------------------------------------------------------------------------
Carrier Sense Multiple Access with Collision Detection (CSMA/CD) is a protocol for medium access in which stations can "sense" the medium and wait until it is quiet before transmitting a message (the "CS" part).
While a station is transmitting, it continues sensing the medium to detect for transmissions from other stations, which would indicate a collision (the "CD" part).
When a collision is detected, the station immediately stops, and reschedules the frame after some time.

Hence, a station would only be aware of a collision if it receives a transmission from another station while it is sending.
If a station finishes a transmission before the first bit of another station's frame arrives, then it will be unaware of the collision and falsely believe it successfully transmitted its prior frame.
For CSMA-CD to properly function, stations must be accurately aware of all collisions so that they can reschedule collided frames.

In this part of the lab, you will complete a CSMA-CD simulator that simply counts collisions: actual collisions versus station-aware collisions.
If the number of actual collisions does not match the number of station-aware collisions, then the stations will inadvertently lose data.
As you'll see, the ability of a station to detect a collision varies depending on different network characteristics and parameters.


%-------------------------------------------------------------------------------------------------
\subsection{Simulated Station Setup}
\label{subsec:station-setup}
%-------------------------------------------------------------------------------------------------
This simulation models the setup of the stations in the following ways:
\begin{itemize}
    \item Assume the stations form a star topology, so all stations are equi-distant
    \item Assume the stations are inter-connected by a hub (not a switch), and hence forms a single broadcast medium
    \item The stations will generate a sequence of arrival times for \textit{N} number of frames (\textit{N} is a configurable parameter). If a frame cannot be sent at a given time (i.e. due to the medium being busy), that frame is rescheduled to the end of the arrival sequence.
\end{itemize}

%-------------------------------------------------------------------------------------------------
\subsection{Simplified CSMA/CD Protocol}
\label{subsec:simplified-csma-cd}
%-------------------------------------------------------------------------------------------------
The actual protocol you will complete is a simplified version of CSMA-CD.
The key modifications, which will make the simulation easier, are as follows:
\begin{itemize}
    \item Upon collision detection, transmission does not stop (i.e. the stations involved will finish transmitting their frames)
    \item When two frames collide, the frame that began transmission earlier in time "loses" (i.e. considered as 1 collision), while the frame that began transmission later in time "wins" (i.e. it can be counted as a success if and only if no subsequent frame collides with it)
    \item Upon collision, no jamming or collision enforcement
    \item No inter-frame spacing (i.e. once the medium is silent, stations with a frame can send)
    \item No backoff mechanism. If a station has a frame to transmit and the medium is busy, it reschedules the frame to the end of its arrival sequence (see above section).
\end{itemize}


%-------------------------------------------------------------------------------------------------
\section{Exerciser}
\label{sec:exercise}
%-------------------------------------------------------------------------------------------------
You can use the exerciser to help test the correctness of your implementation.
The exerciser will run a set of public test cases against your code.
However, note that the tests in the exerciser is not complete and it is your responsibility to test your code to make sure it conforms to the requirement of the lab.
Ensure you have the following files all within the same directory:
\begin{itemize}
    \item \textbf{TODO}: TOOOOOOO DOOOOOOOOOOOOO
\end{itemize}

In terminal, browse to the directory containing the files and type \texttt{ece361-exercise \thelab}.

%-------------------------------------------------------------------------------------------------
\section{Submission}
\label{sec:submission}
%-------------------------------------------------------------------------------------------------
Once you are confident of your implementation, you can run the submission process (which will invoke the exerciser before submitting).
Only one person in the group needs to submit.

From the same directory as where you ran the exerciser, type \texttt{ece361-submit submit \thelab}.

You can then verify the submission by typing \texttt{ece361-submit list \thelab}.

At some point after the lab's due date, private test cases will be run against your submission to calculate your final mark.

%-------------------------------------------------------------------------------------------------
\newpage
\appendix
\section{Appendix}
\label{sec:appendix}
%-------------------------------------------------------------------------------------------------
\textbf{TODO}


\end{document}
