\documentclass[11pt]{article}

\usepackage{../style/ece361}

%-------------------------------------------------------------------------------------------------

% Lab parameters
\def\thelab{5}
\def\datedue{Apr. 9, 2020 @ 11:59 PM}

\useCompactTitle{ECE361 Lab \thelab}{Switching and Routing}

%-------------------------------------------------------------------------------------------------
\begin{document}
% Create title page, and force header and footer onto it
\maketitle \thispagestyle{fancy}

\hfill {\large \textbf{Due \datedue}}
%-------------------------------------------------------------------------------------------------


%-------------------------------------------------------------------------------------------------
\section{Overview}
\label{sec:overview}
%-------------------------------------------------------------------------------------------------
\textbf{TODO}

%-------------------------------------------------------------------------------------------------
\section{Lab Initialization}
\label{sec:lab-init}
%-------------------------------------------------------------------------------------------------
\textbf{TODO: Is there a custom lib for this lab??}

This lab includes starter code that uses a custom-built Python 3 library called \texttt{ece361}.
A Python virtual environment will be created, that will include the library, where you will do all the work in this lab.
We have provided an initialization script for you that will setup this virtual environment, along with the starter code, and the custom library.

\hint{It's highly recommended to update your VM before starting each lab. Open a terminal (in the VM), and type \texttt{ece361-update}}

To run the initialization script for lab \thelab, open a terminal and type \texttt{ece361-lab-init \thelab}.
You should see something similar to Listing~\ref{lst:init-example}.

\begin{lstlisting}[style=ece361shell, caption={Initializing lab \thelab.}, label={lst:init-example}]
ubuntu@ece361:~$ ece361-lab-init (@"\thelab"@)
Finding available UG EECG host...
Warning: Permanently added 'ug251.eecg.utoronto.ca,128.100.13.251' (ECDSA) to the list of known hosts.
Creating working directory for lab (@"\thelab"@) at (@"\btHL /home/ubuntu/lab\thelab"@)
Creating Python3 virtual environment...
Installing libraries...

...

Done. Now run (@"\btHL source /home/ubuntu/lab\thelab/sourceMe"@) to activate the virtual environment.
\end{lstlisting}

The initialization script creates a working directory for this lab, located at \texttt{\textasciitilde/lab\thelab}.
To activate the virtual environment, run the source command as shown in Listing \ref{lst:init-example}.
Once the virtual environment is activated, you should see a \texttt{(.venv)} appear at the beginning of your prompt.

\point{You will need to activate the virtual environment \textbf{in each terminal} that runs code belonging to this lab.}

\warn{The \texttt{ece361} library used in this lab is different than the ones from the previous three labs. Your code from the previous labs will not work in this new virtual environment.}

In your lab \thelab~working directory, you should have the following files:
\begin{itemize}
    \item \texttt{\textbf{TODO FILENAME}}
        \begin{itemize}
            \item \textbf{TODO} File descriptions
        \end{itemize}

    \item \texttt{\textbf{TODO FILENAME}}
        \begin{itemize}
            \item \textbf{TODO} File descriptions
        \end{itemize}
\end{itemize}

\point{If you discover a bug in any of the files provided, please report it on Piazza.}


%-------------------------------------------------------------------------------------------------
\section{Exerciser}
\label{sec:exercise}
%-------------------------------------------------------------------------------------------------
You can use the exerciser to help test the correctness of your implementation.
The exerciser will run a set of public test cases against your code.
However, note that the tests in the exerciser is not complete and it is your responsibility to test your code to make sure it conforms to the requirement of the lab.
Ensure you have the following files all within the same directory:
\begin{itemize}
    \item \textbf{TODO FILENAME}: Your completed \textbf{TODO FILE DESC}.
    \item \textbf{TODO FILENAME}: Your completed \textbf{TODO FILE DESC}.
\end{itemize}

In terminal, browse to the directory containing the files and type \texttt{ece361-exercise \thelab}.


%-------------------------------------------------------------------------------------------------
\section{Submission}
\label{sec:submission}
%-------------------------------------------------------------------------------------------------
Once you are confident of your implementation, you can run the submission process (which will invoke the exerciser before submitting).
Only one person in the group needs to submit.

From the same directory as where you ran the exerciser, type \texttt{ece361-submit submit \thelab}.

You can then verify the submission by typing \texttt{ece361-submit list \thelab}.

At some point after the lab's due date, private test cases will be run against your submission to calculate your final mark.


\end{document}
